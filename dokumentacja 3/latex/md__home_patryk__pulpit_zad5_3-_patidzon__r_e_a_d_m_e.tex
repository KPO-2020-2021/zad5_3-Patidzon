Build by making a build directory (i.\+e. {\ttfamily build/}), run {\ttfamily cmake} in that dir, and then use {\ttfamily make} to build the desired target.

Requirements\+: cmake, gnuplot, doxygen + dot (in graphviz library)

Example\+:


\begin{DoxyCode}{0}
\DoxyCodeLine{> mkdir build \&\& cd build}
\DoxyCodeLine{> cmake .. \#\#\#\# options: -\/DCMAKE\_BUILD\_TYPE=[Debug | Coverage | Release], Debug is default}
\DoxyCodeLine{> make     \#\#\#\# compilation}
\DoxyCodeLine{> ./main   \#\#\#\# main() from app}
\DoxyCodeLine{> make test      \#\#\#\# Compile the tests}
\DoxyCodeLine{> ./unit\_tests -\/s   \#\#\#\# Start the tests, -\/s flag -\/ full description of each case}
\DoxyCodeLine{> make fulltest \#\#\#\# alternative for above, compile \& run tests with full decription}
\DoxyCodeLine{> make coverage  \#\#\#\# Generate a coverage report}
\DoxyCodeLine{> make doc       \#\#\#\# Generate html documentation}
\end{DoxyCode}


Things to remember\+:
\begin{DoxyItemize}
\item changes to C\+Make\+Lists.\+txt in the main folder with new files added, i.\+e.\+: 
\begin{DoxyCode}{0}
\DoxyCodeLine{\textcolor{preprocessor}{\# -\/-\/-\/-\/-\/-\/-\/-\/-\/-\/-\/-\/-\/-\/-\/-\/-\/-\/-\/-\/-\/-\/-\/-\/-\/-\/-\/-\/-\/-\/-\/-\/-\/-\/-\/-\/-\/-\/-\/-\/-\/-\/-\/-\/-\/-\/-\/-\/-\/-\/-\/-\/-\/-\/-\/-\/-\/-\/-\/-\/-\/-\/-\/-\/-\/-\/-\/-\/-\/-\/-\/-\/-\/-\/-\/-\/-\/-\/-\/-\/}}
\DoxyCodeLine{\textcolor{preprocessor}{\#                         Locate files (change as needed).}}
\DoxyCodeLine{\textcolor{preprocessor}{\# -\/-\/-\/-\/-\/-\/-\/-\/-\/-\/-\/-\/-\/-\/-\/-\/-\/-\/-\/-\/-\/-\/-\/-\/-\/-\/-\/-\/-\/-\/-\/-\/-\/-\/-\/-\/-\/-\/-\/-\/-\/-\/-\/-\/-\/-\/-\/-\/-\/-\/-\/-\/-\/-\/-\/-\/-\/-\/-\/-\/-\/-\/-\/-\/-\/-\/-\/-\/-\/-\/-\/-\/-\/-\/-\/-\/-\/-\/-\/-\/}}
\DoxyCodeLine{set(SOURCES          \# All .cpp files in src/}
\DoxyCodeLine{    src/lacze\_do\_gnuplota}
\DoxyCodeLine{    src/Matrix2x2.cpp}
\DoxyCodeLine{    src/Rectangle .cpp}
\DoxyCodeLine{    src/Vector2D.cpp \# etc.}
\DoxyCodeLine{)}
\DoxyCodeLine{set(TESTFILES        \textcolor{preprocessor}{\# All .cpp files in tests/}}
\DoxyCodeLine{    Vector2D.cpp}
\DoxyCodeLine{    Matrix2x2.cpp}
\DoxyCodeLine{    Rectangle.cpp \# etc.}
\DoxyCodeLine{)}
\DoxyCodeLine{set(LIBRARY\_NAME zadX)  \textcolor{preprocessor}{\# Default name for the library built from src}\textcolor{comment}{/*.cpp (change if you wish)}}
\end{DoxyCode}

\item changes to \mbox{\hyperlink{tests_2_c_make_lists_8txt}{tests/\+C\+Make\+Lists.\+txt}} (in tests subfolder) with new files added, i.\+e.\+: 
\begin{DoxyCode}{0}
\DoxyCodeLine{\textcolor{preprocessor}{\# List all files containing tests. (Change as needed)}}
\DoxyCodeLine{set(TESTFILES        \# All .cpp files in tests/}
\DoxyCodeLine{    \mbox{\hyperlink{app_2main_8cpp_ae66f6b31b5ad750f1fe042a706a4e3d4}{main}}.cpp}
\DoxyCodeLine{    test\_Wektor2D.cpp}
\DoxyCodeLine{    test\_Macierz2x2.cpp}
\DoxyCodeLine{    test\_Prostokat.cpp \# etc.}
\DoxyCodeLine{)}
\end{DoxyCode}

\end{DoxyItemize}

The {\ttfamily main.\+cpp} in the folder {\ttfamily tests} is needed. Only there we define {\ttfamily \#define D\+O\+C\+T\+E\+S\+T\+\_\+\+C\+O\+N\+F\+I\+G\+\_\+\+I\+M\+P\+L\+E\+M\+E\+N\+T\+\_\+\+W\+I\+T\+H\+\_\+\+M\+A\+IN}.

The main loop of the program is in the {\ttfamily app} folder.

Wazne informacje\+: rotory drona nie kreca sie podczas lotu i wznoszenia Można maksymalnie dodać 9 przeszkod, ale usuniete przeszkody nie zwalniaja limitu(nie wiedzialem jak zapisywac przeszkody do wczesniejszych plikow). Poczatkowe przeszkody trzeba ustawic recznie(przerobienie konstruktorow zajeloby mi za duzo czasu ) drony i przeszkody sa na osobnych listach(wybralem uproszcenie) Nowo tworzone preszkody moga powstac na istniejacych obiektach(w opisie zadania nie bylo o tym mowy(chyba), a ze wzgledu na skalowanie przeszkod i problem z ponownym wykorzystaniem plikow zapisu nie zdecydowalem sie na to ) Przy drugim przemieszczeniu sie drona laduje on polowa korpusu \char`\"{}w ziemi\char`\"{} blad ten sie nie poglebia(kazde kolejne ladowanie konczy sie tak samo) i nie udalo mi sie znalezc jego przyczyny. 